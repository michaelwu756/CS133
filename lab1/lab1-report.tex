\documentclass[12pt]{article}
\begin{document}
\title{Computer Science 133, Lab 1}
\date{January 24th, 2019}
\author{Michael Wu\\UID: 404751542}
\maketitle

\section{Speedup Summary}

The following table summarizes the results of my implementation.
\begin{center}
    \begin{tabular}{c|c|c|c}
        Problem Size & \(1024^3\) & \(2048^3\) & \(4096^3\)\\
        \hline
        Sequential Speed (GFlops) & 0.595169 & 0.283576 & 0.195333\\
        Parallel Speed (GFlops) & 19.3156 & 19.4528 & 17.8648\\
        Parallel Speedup & 32.45x & 68.60x & 91.46x\\
        Parallel-Blocked Speed (GFlops)& 15.3914 & 17.6912 & 21.0508\\
        Parallel-Blocked Speedup & 25.86x & 62.39x & 107.77x
    \end{tabular}
\end{center}
The speedup differences are mainly due to the sequential version performing
worse as the problem size increases, since there are more cache misses in the
sequential version. The non-blocked parallel implementation began to drop
off slightly as the problem size increased due to cache misses as well. The
blocked version seemed to increase in performance as the problem size increased,
since blocking works better when used on a larger array.

\section{Parallel Blocked Optimizations}

There seemed to be a wide range of block size values that could lead to good performance,
as long as they were not too large or too small. The following table summarizes the performance
of my blocked parallel implementation in relation to block size on the \(4096^3\) problem size.
I chose to use a block size of 256 since it led to good performance in all three problem sizes
of \(1024^3\), \(2048^3\), and \(4096^3\).
\begin{center}
    \begin{tabular} {c|c}
        Block Size & Speed (GFlops)\\
        \hline
        32 & 16.6554\\
        256 & 21.0924\\
        512 & 20.5402\\
        1024 & 21.0501\\
        2048 & 18.2274
    \end{tabular}
\end{center}
Additional optimizations I performed were loop unrolling and transposing the \(b\) matrix so that the innermost
loop traversed the rows of \(b\) instead of the columns. Loop unrolling gave me approximately twice the speed,
and transposing the matrix gave me approximately ten times the speed. Note that loop unrolling led to small
rounding errors on the order of \(10^{-7}\) as the array size increased, as the floating point addition might have
been performed in a different order.

\section{Parallel Blocked Thread Performance}

The \texttt{m5.2xlarge} instance supports up to 8 threads, since there are 8 cores on the machine.
The following summarizes the scalability of my blocked implementation with different thread numbers on the
\(4096^3\) problem size.
\begin{center}
    \begin{tabular} {c|c}
        Threads & Speed (GFlops)\\
        \hline
        1 & 4.70493\\
        2 & 9.71357\\
        4 & 19.1020\\
        8 & 21.0384
    \end{tabular}
\end{center}

\section{Results}

In my implementation, it seems that the blocked parallel implementation outscales the regular parallel implementation
as the array size increases. For small arrays it is faster to use the regular parallel implementation. Note that
as array size increases, the speedup over the sequential version increases as well because of cache efficiency. Some
of this has no relation to parallelism, because even with a single thread the optimizations can lead to a speed of
4.70493 GFlops in the blocked implementation. With additional threads, the blocked implementation can become over five
times faster.

\section{Challenges}

Two main challenges I faced were related to loop unrolling and testing my implementations. Loop unrolling caused small
rounding errors that led me to believe my implementation was incorrect. I only was able to fix this after reading on Piazza
that the TAs would raise the error threshold to the order of \(10^{-5}\), so these small rounding errors would be acceptable.
Testing my implementation was also a challenge because for most problem sizes the parallel implementation performed better than
the blocked parallel implementation. I was considering submitting my parallel implementation since I was confused on which
problem size my implementation would be tested on. There were so many tuning parameters that I wasted a lot of time fiddling with.
In the end I chose to optimize for the \(4096^3\) problem size, as that is what the TAs gave their benchmarks for.

\end{document}