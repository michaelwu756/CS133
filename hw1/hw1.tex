\documentclass[12pt]{article}
\begin{document}
\title{Computer Science 133, Homework 1}
\date{January 17th, 2019}
\author{Michael Wu\\UID: 404751542}
\maketitle

\section*{Problem 1}

My phone has 4 cores, and the processor is a Qualcomm Snapdragon 820 MSM8996. My computer has 4 cores, and
the processor is an Intel Core i7-4700MQ CPU @ 2.40GHz.

\section*{Problem 2}

Dennard Scaling refers to the phenomenon of the power density of a processor staying constant as transistor
size decreases. So smaller transistors would consume the same amount of power on a chip of a given size.
This was partially due to voltage levels across the transistors decreasing. It broke down because current
leakage would occur as the voltage became smaller, so voltage could no longer scale. As a side effect,
processor frequencies could not be increased without increasing power consumption by too much.

\section*{Problem 3}

\begin{center}
    \begin{tabular}{c|c}
        System & Efficiency (Rmax/Power)\\
        \hline
        Summit & 14.6683\\
        Sierra & 12.72385\\
        Sunway TaihuLight & 6.051304\\
        Tianhe-2A & 3.324559\\
        Piz Daint & 8.905201\\
        Trinity & 2.660161\\
        AI Bridging Cloud Infrastructure & 12.05579\\
        SuperMUC-NG & N/A\\
        Titan & 2.14277\\
        Sequoia & 2.176578
    \end{tabular}
\end{center}
The top 3 most power efficient computers are Summit, Sierra, and AI Bridging Cloud Infrastructure.

\section*{Problem 4}

An example of a application in which we don't know the number of tasks ahead of time is an event queue.
In an event queue, we must wait for tasks to come in during the lifetime of the event queue.

\section*{Problem 5}

We can write the code as follows.
\begin{verbatim}
i=1;
r=d[0];
for (i=0; i<=SIZE; i++) {
  r*=v[i];
}
\end{verbatim}
Then each loop iteration only needs to load then multiply. There is no data dependency, and this loop
produces the same result as the original while reducing the initiation interval to 1.

\end{document}